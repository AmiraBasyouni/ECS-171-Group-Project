% define document type, font-size, and paper dimensions
\documentclass[11pt, letterpaper]{article}
% set up image package & path
\usepackage{graphicx}
% subcaption for multi-figures
\usepackage{subcaption}
% adds a header onto every page
\usepackage{fancyhdr}
% adds indent to paragraph after section header
\usepackage{indentfirst}

% set up page style & graphic path
\pagestyle{fancy}
\graphicspath{{./images/}}

% Set up title page
\title{ECS 171 Group Project}
\author{Amira Basyouni, Alexis Lydon, Calvin Chen, Ryan Yu, Tianming Tan}
\date{\today}

% set up header
\fancyhf{} % clear all header and footer fields
\lhead{ECS 171 Group Project} % left header
\rhead{\thepage} % right header
\fancyhfoffset[LR]{1cm} % increase left, right margin by 1cm

\begin{document}
    \maketitle
    \newpage

    \section*{Problem Statement}
    Sleep, as a student, many of us don't get enough of it. We've heard it 
    affects our health, but in what ways? How much is enough sleep and why 
    should we care? After analyzing our data, we can hopefully understand 
    why this is important and improve our sleep quality. We will be able to 
    narrow down the factors that affect our sleep, and build habits that can 
    mitigate and reduce the negative effects sleep deprivation has on our 
    health. Through the use of Machine Learning models, we can create 
    predictions on what factors would lead to better or worse sleep, 
    and predict if a person would have high or low sleep quality based 
    on these factors. This can allow us to gain more insight into how to 
    improve our sleep quality. Not only will this help students like us 
    better manage our sleep, but it will also help us as a community of 
    diverse occupations to get a good night's rest.

    \section{Dataset}

    \section{Goals}

    \section{Project Timeline}

\end{document}