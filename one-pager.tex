% define document type, font-size, and paper dimensions
\documentclass[11pt, letterpaper]{article}
% set up image package & path
\usepackage{graphicx}
% subcaption for multi-figures
\usepackage{subcaption}
% adds a header onto every page
\usepackage{fancyhdr}

% set up page style & graphic path
\pagestyle{fancy}
\graphicspath{{./images/}}

% Set up title page
\title{ECS 171 Group Project}
\author{Amira Basyouni, Alexis Lydon, Calvin Chen, Ryan Yu, Tianming Tan}
\date{\today}

% set up header
\fancyhf{} % clear all header and footer fields
\lhead{ECS 171 Group Project} % left header
\rhead{\thepage} % right header
\fancyhfoffset[LR]{1cm} % increase left, right margin by 1cm

\begin{document}
    \maketitle
    \newpage

    \section{Introduction}

    \section{Dataset}
    The dataset being used in this project is the Sleep Health and Lifestyle dataset by Laksika Tharmalingam on Kaggle. 
    The dataset comes in the form of a CSV file and contains 400 rows and 13 columns, encompassing various sleep and lifestyle variables. 
    These include gender, age, occupation, sleep duration, sleep quality, physical activity, stress levels, BMI category, blood pressure, heart rate, daily steps, 
    and sleep disorder status. We chose this dataset because, as students, we can relate to the common sleep issues many of us face. Our project focuses on investigating 
    the relationship between sleep and health, making a dataset on sleep and health the most suitable choice. A limitation of the dataset is its synthetic nature, generated 
    artificially rather than from real-world observations. While it may lack some real-world nuances, high-quality synthetic data can effectively train and test machine learning 
    models in this context.
    \section{Goals}

    \section{Project Timeline}

\end{document}